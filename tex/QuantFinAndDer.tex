\documentclass[oneside,titlepage,headinclude,12pt,a4paper,BCOR5mm,footinclude]{book}
\usepackage{pdfpages}
\usepackage[T1]{fontenc}
\usepackage{mathtools}
\usepackage{amsmath}
\usepackage{amsfonts}
\usepackage{amsthm}
\usepackage{enumitem} 
\usepackage{cancel}
\renewcommand*\rmdefault{ppl}
\setlength{\parindent}{0em}
\setlength{\parskip}{0.5em}
%\theoremstyle{definition}
\newtheoremstyle{defn}
  {\topsep}
  {\topsep}
  {\normalfont}
  {0pt}
  {\bfseries}
  {.}
  {5pt plus 1pt minus 1pt}
  {}
\theoremstyle{defn}
\newtheorem{defn}{Definition}
\newtheorem{example}{Example}

\newcommand{\eexp}{\mathrm{e}}
\newcommand{\de}{\partial}

\title{Quantitative Finance and Derivatives}

\begin{document}

\maketitle
\tableofcontents

\chapter{Models for asset pricing}

\section{Underlying processes}

  \subsection{Stochastic Processes}

  Stochastic  processes are  collections  of random  variables representing  the
  evolution of some system over time.  Stochastic processes can have discrete or
  continuous values, and can evolve over discrete or continuous time.

  We have four different possible situations:
  \begin{enumerate}
    \item \(X(n,\omega) : \mathbb{N} \times (\Omega,\mathcal{F},\mathbb{P}) \to \text{a subset of } \mathbb{Z}:\) discrete time, discrete values. For example: a \textit{random walk}.
    \item \(X(n,\omega) : \mathbb{N} \times (\Omega,\mathcal{F},\mathbb{P}) \to \text{a subset of } \mathbb{R}:\) discrete time, continuous values.
    \item \(X(n,\omega) : \mathbb{R}^+ \times (\Omega,\mathcal{F},\mathbb{P}) \to \text{a subset of } \mathbb{Z}:\) continuous time, discrete values. For example: a \textit{Poisson process}.
    \item \(X(n,\omega) : \mathbb{R}^+ \times (\Omega,\mathcal{F},\mathbb{P}) \to \text{a subset of } \mathbb{R}:\) continuous time, continuous values. For example: a \textit{Brownian motion}.
  \end{enumerate}

  \subsubsection{From random walk to Brownian motion}

  Let   $$(X_n)_{n   \geq    0}$$   be   a   stochastic    process   such   that
  \(X_i\),   for   any   \(i\),   can  take   value   \(1\)   with   probability
  \(\mathbb{P}(X_i=1)=\frac{1}{2}\),     and     \(-1\)     with     probability
  \(\mathbb{P}(X_i=-1)=\frac{1}{2}\). Let then \(S_n = X_1 + X_2 + \ldots +X_n\)
  be the  position you  are at  on a  line after  \(n\) steps.  Informally, each
  \(X_i\) "moves" you randomly one step to the right or one step to the left. We
  want  to know  expectation  and variance  of this  process.  Let's assume  the
  \((X_n)_{n \geq  0}\) to be \textit{independent  and identically distributed}.
  We can now assert the following:

  \[
    E\left[X_i\right] = (+1) \cdot \frac{1}{2} + (-1) \cdot \frac{1}{2} = 0
  \]
    
  \[
    E\left[X_i^2\right] = (+1)^2 \cdot \frac{1}{2} + (-1)^2 \cdot \frac{1}{2} = 1
  \]

  Hence,

  \[
    E\left[S_n\right] = \sum_{i=1}^n E\left[X_i\right] = 0
  \]

  \[
    Var\left[S_n\right] = \sum_{i=1}^n E\left[X_i^2\right] = n
  \]

  As \(n\)  approaches infinity,  the conditions  for the  \textit{Central Limit
  Theorem} hold: \(X_i\)s are \textit{i.i.d.}, so we can apply

  \[
    \frac{S_n - n \cdot E\left[X_i\right]}{\sqrt{n \cdot Var\left[X_i\right]}}
    = \frac{S_n}{\sqrt{n}} \underset{n \to \infty}{\sim} \mathcal{N}(0,1)
  \]

  Consider now a non-unitary  time. Suppose we move in time  steps of \(\delta >
  0\) and in space steps of \(\sqrt{\delta}\), and let's consider the process in
  the interval $[0,t],\ t\in\mathbb{R}^+$. Then,

  \[
    S_t = \sum_{i=0}^{\left\lfloor \frac{t}{\delta} \right\rfloor} X_i
  \]

  where     $X_i$     moves     by     $\pm\sqrt\delta$     with     probability
  $\mathbb{P}=\frac{1}{2}$. Then, $E\left[S_t\right]=0$ and

  \[
    Var[X_i] = E^2[X_i] - E[X_i^2] = E^2[X_i] = (+\sqrt{\delta})^2 \cdot \frac{1}{2} + (-\sqrt{\delta})^2 \cdot \frac{1}{2} 
      = \delta
  \]

  \[
    Var[S_t] = \frac{t}{\delta} \cdot \delta = t
  \]

  Let's now apply \textit{Central Limit Theorem}:

  \[
    \frac{S_t - \frac{t}{\delta} E[X_i]}{\sqrt{\frac{t}{\delta} Var[X_i]}}
    = \frac{S_t}{\sqrt{t}} \underset{t \to \infty}{\sim} \mathcal{N}(0,1).
  \]

  So, $S_t \sim \mathcal{N}(0,t)$: this is the \textit{Brownian motion}.

  \begin{defn} 
    \textbf{(Brownian motion).} The stochastic process 

    \[
      (W_t)_{t\in\mathbb{R}^+} : \mathbb{R}^+ \times (\Omega, (\mathcal{F}_t)_{t \geq 0}, \mathbb{P}) \to \mathbb{R}
    \]

    is a Brownian motion if and only if

    \begin{enumerate}[label=(\Roman*)]
      \item $W_0 = 0$,
      \item it is continuous,
      \item  has  stationary  increments:  the distribution  doesn't  depend  on
             initial time, only on waiting time.
        
        $$\forall t > s,\ W_t - W_s \sim \mathcal{N}(0,t-s)$$

      \item has independent increments over disjoint intervals: 
        
        $$\forall q < r < s < t,\ (W_r-W_q) \perp (W_t-W_s).$$
    \end{enumerate}
  \end{defn}

  %\begin{enumerate}
  %  \item From Random Walk to Brownian Motion
  %  \item Brownian motion definition
  %  \item Classifications
  %\end{enumerate}

  \begin{defn}\textbf{(Classes of Brownian motions.)}

  \begin{enumerate}[label=(\Roman*)]
    \item \textit{Standard Brownian motion}, or \textit{Wiener Process}.
    \item \textit{Arithmetic Brownian motion}, or \textit{Bachelier Model}.
      
      \[
        dp_t = p_{t+h} - p_t = \mu dt + \sigma dW_t
      \]

      Where    $\mu$    is    called   \textit{drift},    and    $\sigma$    the
      \textit{volatility}. This is  a \textit{stochastic differential equation},
      \textit{SDE} for short: while the $\mu dt$ part is \textit{deterministic},
      the $\sigma  dW_t$ contains  a \textit{stochastic}  component, which  is a
      standard Brownian motion. This kind of  equations provides a model for the
      change of the  price $p_t$ over the infinitesimal time  increment from $t$
      to $t+h$. The price variation could be equivalently modeled this way:

      \[
        p_t - p_0 = \mu(t-0) + \sigma(W_t - W_0) = \mu t + \sigma W_t
      \]
      
      Noting that 

      \[ 
        E[W_t]  =  0
        \quad \text{and} \quad
        Var[\sigma W_t]  =  E[(\sigma W_t)^2]  = \sigma^2 E[W_t^2] =  \sigma^2 t,
      \]

      we have that, in Bachelier's model, the price  $p_t = \mu t + \sigma W_t +
      p_0$ is a \textit{random variable} distributed like

      \[
        p_t \sim \mathcal{N}(p_0 + \mu t + E[W_t], \sigma^2 t) \equiv
          \mathcal{N}(p_0 + \mu t, \sigma^2 t).
      \]

    \item \textit{Geometric Brownian motion}, or \textit{Black-Scholes model}.

      \[
        dp_t = \mu p_t dt + \sigma p_t dW_t
      \]

      This model is similar to Bachelier's, but acts on a multiplicative instead
      of additive principle. For this model, we will need to find a solution and
      its distribution.

  \end{enumerate}
  \end{defn}

  \subsection{It\=o Formula}

  Suppose a model for  an underlying asset price dynamics is  given: we know the
  form of $dp_t$;  we'll assume Black-Scholes. It is now  natural to assume that
  the price  of a  derivative on this  asset is a  \textit{function}, let  it be
  $f(p_t)$, of the asset price. How can we get a \textit{stochastic differential
  equation} that models the derivative price variation?

  Let's evaluate the \textit{Taylor expansion} for $f(p_t)$ like in a deterministic setting.
  
  \[
    f(x) \approx f(x_0) + f'(x_0)(x-x_0) + \frac{1}{2} f''(x_0)(x-x_0)^2 + \varepsilon
  \]

  where $\varepsilon$  is a  remainder that  goes quickly to  zero. We  will now
  discard all  terms in the  approximation that go to  0 faster than  $dt$ does,
  thus truncating the approximation to to the first order. Let now $x=p_{t+dt}$,
  $x_0 = p_t$ and $dp_t = p_{t+dt} - p_t$
  
  \[
    f(p_{t+dt}) \approx f(p_t) + f'(p_t)(dp_t) + \frac{1}{2} f''(p_t)(dp_t)^2 + \varepsilon
  \]
  
  and substitute the increment $dp_t$ with the Black-Scholes model:

  \begin{align*}
    f(p_{t+dt}) = & f(p_t) + f'(p_t)(\mu p_t dt + \sigma p_t dW_t) +\\ 
                  & \frac{1}{2}f''(p_t)(\mu^2 p_t^2 (dt)^2 + \sigma^2 p_t^2 (dW_t)^2 + 2 \mu \sigma p_t^2 dtdW_t) 
  \end{align*}

  Since $E[(dW_t)^2] = dt$, we can  assume $dW_t = \sqrt{dt} = (dt)^\frac{1}{2}$
  to have order $\frac{1}{2}$. Let's analyze the  orders of all the terms in the
  approximation:
  \begin{itemize}
    \item $\mu p_t dt$: First order: keep it.
    \item $\sigma p_t dW_t$: Order $\frac{1}{2}$: keep it.
    \item $\mu^2 p_t^2 (dt)^2$: Second order: discard it.
    \item $\sigma^2 p_t^2 (dW_t)^2$: First order, because $(dW_t)^2 = dt$: keep it.
    \item $2 \mu \sigma p_t^2 dtdW_t$: Order $\frac{3}{2}$ because of $dtdW_t$: discard it.
  \end{itemize}
  The truncated approximation now states that

  \[
    f(p_{t+dt}) = f(p_t) + f'(p_t)(\mu p_t dt + \sigma p_t dW_t) + 
    \frac{1}{2} f''(p_t)(\sigma^2 p_t^2 dt). 
  \]

  The second derivative term of this equation is called \textit{It\=o correction
  term}.  The derivative  of  the function  with  respect to  time  can then  be
  computed this way:

  \[
    df(p_t) = f(p_{t+dt}) - f(p_t) = f'(p_t)(\mu p_t dt + \sigma p_t dW_t) + 
    \frac{1}{2} f''(p_t)(\sigma^2 p_t^2 dt)
  \]
  
  Let's now  separate the deterministic terms  from the stochastic terms,  so we
  can identify a \textit{drift} and a \textit{volatility} for the model.
  
  \[
    df(p_t) = \left[ f'(p_t)\mu p_t + \frac{1}{2}f''(p_t)\sigma^2 p_t^2\right]dt
      + f'(p_t)\sigma p_t dW_t
  \]

  \begin{example}
    Let  $f(p_t) =  \ln(p_t)$  be  the price  of  a  derivative instrument  with
    underlying  price  $p_t$;  the  dynamic   for  the  underlying  follows  the
    Black-Scholes model. Apply It\=o's formula to this derivative:

    \begin{align*}
      d\ln(p_t) = & \left[ \frac{1}{p_t} \mu p_t + \frac{1}{2}\left( 
        -\frac{1}{p_t^2}\sigma^2 p_t^2\right)\right]dt +
        \frac{1}{p_t}\sigma p_t dW_t = \\
        = & \left[\mu - \frac{1}{2} \sigma^2 \right]dt + \sigma dW_t
    \end{align*}

    Considering the time interval $[0,t]$, we observe that
    
    \[
      d\ln(p_t) = \ln(p_t) - \ln(p_0) = \left( \mu - \frac{1}{2}\sigma^2 \right)
        (t-0) + \sigma (W_t - W_0)
    \]
    
    and then
    
    \[
      \ln\left(\frac{p_t}{p_0}\right) = \left(\mu -\frac{1}{2}\sigma^2\right)t + \sigma W_t
      \iff
      \frac{p_t}{p_0} = \mathrm{e}^{\left(\mu - \frac{1}{2}\sigma^2\right)t + \sigma W_t}
    \]
    
    which yields the solution
    
    \[
      p_t = p_0 \mathrm{e}^{\left(\mu - \frac{1}{2}\sigma^2\right)t + \sigma W_t}.
    \]
  \end{example}

  \begin{example}
    Black-Scholes model assumes the \textit{spot risk-free interest rate} $r$ to
    be constant and independent of  maturity. Consider \textit{discounting} as a
    function of time and asset price:

    \[
      f(t;p_t) = \eexp^{-rt}p_t
    \]

    The general It\=o formula for a function of such a form is

    \begin{align*}
      d[f(t;p_t)] = & 
        \frac{\partial f}{\partial p_t} dp_t + 
        \frac{\partial f}{\partial t} dt + 
        \frac{1}{2} \frac{\partial^2 f}{\partial p_t^2} (dp_t)^2 +
        \underbrace{\cancel{\frac{1}{2} \frac{\partial^2 f}{\partial t^2} (dt)^2}}_\text{order 2} +
        \underbrace{\cancel{\frac{1}{2} \frac{\partial^2 f}{\partial p_t \partial t} 2dtdp_t}}_\text{order 3/2}
        \\
        = & 
        \frac{\partial f}{\partial p_t} dp_t + 
        \frac{\partial f}{\partial t} dt + 
        \frac{1}{2} \frac{\partial^2 f}{\partial p_t^2} \sigma^2 p_t^2dt
    \end{align*}

    For the discounting function, this means

    \begin{align*}
      d[e^{-rt}p_t] = & \eexp^{-rt} \cdot 1 \cdot dp_t + -r\eexp^{-rt}p_tdt +
        \cancelto{0}{\frac{1}{2} \cdot 0 \cdot \sigma^2 p_t^2 dt}
        \\
        = & \eexp^{-rt}\left( \mu p_t dt + \sigma p_t dW_t -r p_t dt \right)
        \\
        = & (\mu - r) \expe^{-rt} p_tdt + \expe^{-rt} p_t\sigma dW_t
        \\
      d\tilde{p}_t = & (\mu - r)\tilde{p}_tdt + \sigma \tilde{p}_t dW_t
    \end{align*}

    Hence, the distribution for a discounted asset price follows Black-Scholes model:

    \begin{align*}
      \tilde{p}_t = \tilde{p}_0 \eexp^{(\mu - r - \frac{\sigma^2}{2})t + \sigma W_t}
    \end{align*}

    Note that the  deterministic and stochastic parts were  grouped together, to
    underline the \textit{risk factor}.
  \end{example}

  \subsection{Black-Scholes PDE for option pricing} 

  In the context of option pricing, Black-Scholes model assumes the following:

  \begin{itemize}
    \item $p_t$ is the underlying asset price at time $t$,
    \item $r$ is a constant, risk-free interest rate,
    \item There are no transaction costs, no taxes, and no arbitrage opportunity,
    \item The market is liquid, and so all the instruments,
    \item $f(t;p_t)$ is the option price.
  \end{itemize}

  By It\=o's lemma, we get that the option price is

  \[
    df(t;p_t) = \left[ \frac{\partial f}{\partial t}} + \frac{\partial f}{\partial p_t} \mu p_t +
      \frac{1}{2}\frac{\partial^2 f}{\partial p_t^2} \sigma^2 p_t^2 \right] dt +
      \frac{\partial f}{\partial p_t} \sigma p_t dW_t
  \]

  Construct a \textit{locally} risk-free portfolio, $\Pi_t$, such that

  \[
    \Pi_t = \left\{ 
      \begin{array}{cl}
        -1 & \text{positions in options (short)} \\
        \Delta_t \equiv \frac{\partial f}{\partial p_t} & \text{positions in underlying (long)}
      \end{array}
  \]

  and study  the dynamics of  the portfolio value  by multiplying the  number of
  positions by the dynamics for each kind of instrument (option and asset).

  \begin{align*}
    d\Pi_t = & -1 \cdot df(t;p_t) + \frac{\partial f}{\partial p_t} dp_t
    \\ = & \underbrace{- \left[
        \frac{\partial f}{\partial t} + \frac{\partial f}{\partial p_t} \mu p_t +
        \frac{1}{2} \frac{\partial^2 f}{\partial p_t^2} \sigma^2 p_t^2 \right] dt -
        \frac{\partial f}{\partial p_t}\sigma p_t dW_t}_\text{option} +
      \underbrace{
        \frac{\partial f}{\partial p_t} \left( \mu p_t dt + \sigma p_t dW_t \right)
      }_\text{asset}
    \\ = & - \left[
        \frac{\partial f}{\partial t} + 
        \frac{1}{2} \frac{\partial^2 f}{\partial p_t^2} \sigma^2 p_t^2 \right] dt -
        \cancel{\frac{\partial f}{\partial p_t} \mu p_t dt} -
        \cancel{\frac{\partial f}{\partial p_t}\sigma p_t dW_t}
        + \cancel{\frac{\partial f}{\partial p_t} \mu p_t dt} + 
        \cancel{\frac{\partial f}{\partial p_t} \sigma p_t dW_t}
    \\ = & - \left( \frac{\partial f}{\partial t} + 
      \frac{1}{2}\frac{\partial^2 f}{\partial p_t^2} \sigma^2 p_t^2 \right) dt
  \end{align*}

  Having removed  the Brownian motion,  we are left  without any risky  term. We
  impose now the  \textit{no arbitrage assumption}, stating that  a portfolio is
  risk-free if  and only  if its dynamics  is the  same of a  bond, that  is, it
  accrues interest at a constant (by assumption) rate over time.

  \[
    d\Pi_t \equiv - \left( \frac{\partial f}{\partial t} + 
      \frac{1}{2}\frac{\partial^2 f}{\partial p_t^2} \sigma^2 p_t^2 \right) dt 
      \stackrel{\text{NAA}}{=} r\Pi_t dt
  \]

  \[
    -\frac{\de f}{\de t} -\frac{1}{2}\frac{\de^2f}{\de p_t^2} \sigma^2 p_t^2 =
    -rf(t;p_t) + \frac{\de f}{\de p_t} rp_t
  \]

  Finally, we obtain the Black-Scholes PDE by rearranging.

  \[
    rf(t;p_t) = \frac{\de f}{\de t} + \frac{\de f}{\de p_t} rp_t + \frac{1}{2}\frac{\de^2 f}{\de p_t^2} \sigma^2 p_t^2
  \]

  The objective  is to  establish the  \textit{fair} (or  \textit{no arbitrage})
  price of an option today, that  is, $f(0;p_0)$. A \textit{final condition} can
  be imposed:  $f(T,p_T)$, which  is the \textit{option  price at  maturity} or,
  equivalently, the \textit{payoff}, which is known.

  \[
    f(T;p_T) \stackrel{e.g.}{=}
    \begin{array}{ll}
      (P_T - K)^+ = \max \{ P_T - K, 0 \} & \text{European call option} \\
      (K - P_T)^+ = \max \{ K - P_T, 0 \} & \text{European put option}
    \end{array}
  \]

  European  options  satisfy  Black-Scholes   assumptions:  the  payoff  depends
  \textit{only}  on the  price of  the underlying  at time  $T$, and  it is  not
  \textit{path dependent}.  The drift term $\mu$  does not appear in  the payoff
  function $f(t;p_t)$;  this means that  the Black-Scholes option  price doesn't
  depend on  it. The drift  term is  strongly linked to  investor's \textit{risk
  aversion}: this means the option can  be priced \textit{as if} the investor is
  \textit{risk neutral}.

  A \textit{risk neutral} valuation of current option price, given payoff $f(T,p_T)$, is

  \[
    f(0;p_0) = \tilde{\mathbb{E}} \left[ e^{-rT} f(T;p_T)\right]
  \]

  Where $\tilde{\mathbb{E}}$  is the  expectation according to  the \textit{risk
  neutral  probability}   $\tilde{\mathbb{P}}$.  We  now  have,   in  fact,  two
  probability  spaces: the  first is  the \textit{historical  probability space}
  $(\Omega,\mathcal{F},\mathbb{P})$,  the  second  is the  \textit{risk  neutral
  probability  space}   $(\Omega,  \mathcal{F},   \tilde{\mathbb{P}})  \supseteq
  (\Omega,\mathcal{F},\mathbb{P})$. To  calculate the option price  at any given
  time $t : 0 <  t < T$ the information in the filtration up  to time $t$ can be
  used:

  \[
    f(t;p_0) = \tilde{\mathbb{E}} \left[ e^{-r(T-t)} f(T;p_t) \left| \mathcal{F}_t\right]
  \]

\end{document}
