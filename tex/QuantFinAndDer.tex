\documentclass[twoside,openright,titlepage,headinclude,12pt,a4paper,BCOR5mm,footinclude]{book}
\usepackage{pdfpages}
\usepackage[T1]{fontenc}
\usepackage{mathtools}
\usepackage{amsmath}
\usepackage{amsfonts}
\usepackage{amsthm}
\theoremstyle{definition}
\renewcommand*\rmdefault{ppl}
\setlength{\parindent}{0em}
\setlength{\parskip}{0.5em}
\newtheorem{defn}{Definition}

\title{Quantitative Finance and Derivatives}

\begin{document}

\maketitle
\tableofcontents

\chapter{Models for asset pricing}

\section{Underlying processes}

  \subsection{Stochastic Processes}

  Stochastic  processes are  collections  of random  variables representing  the
  evolution of some system over time.  Stochastic processes can have discrete or
  continuous values, and can evolve over discrete or continuous time.

  We have four different possible situations:
  \begin{enumerate}
    \item \(X(n,\omega) : \mathbb{N} \times (\Omega,\mathcal{F},\mathbb{P}) \to \text{a subset of } \mathbb{Z}\):

      discrete time, discrete values. For example: a \textit{random walk}.
    \item \(X(n,\omega) : \mathbb{N} \times (\Omega,\mathcal{F},\mathbb{P}) \to \text{a subset of } \mathbb{R}\):

      discrete time, continuous values.
    \item \(X(n,\omega) : \mathbb{R}^+ \times (\Omega,\mathcal{F},\mathbb{P}) \to \text{a subset of } \mathbb{Z}\):

      continuous time, discrete values. For example: a \textit{Poisson process}.
    \item \(X(n,\omega) : \mathbb{R}^+ \times (\Omega,\mathcal{F},\mathbb{P}) \to \text{a subset of } \mathbb{R}\):

      continuous time, continuous values. For example: a \textit{Brownian motion}.
  \end{enumerate}

  \subsubsection{From random walk to Brownian motion}

  Let   $(X\_n)\_{n   \geq    0}$   be   a   stochastic    process   such   that
  \(X_i\),   for   any   \(i\),   can  take   value   \(1\)   with   probability
  \(\mathbb{P}(X_i=1)=\frac{1}{2}\),     and     \(-1\)     with     probability
  \(\mathbb{P}(X_i=-1)=\frac{1}{2}\). Let then \(S_n = X_1 + X_2 + \ldots +X_n\)
  be the  position you  are at  on a  line after  \(n\) steps.  Informally, each
  \(X_i\) "moves" you randomly one step to the right or one step to the left. We
  want  to know  expectation  and variance  of this  process.  Let's assume  the
  \((X_n)_{n \geq  0}\) to be \textit{independent  and identically distributed}.
  We can now assert the following:

  \[
    E\left[X_i\right] = (+1) \cdot \frac{1}{2} + (-1) \cdot \frac{1}{2} = 0
  \]
    
  \[
    Var\left[S_n\right] \to E\left[X_i^2\right] = (+1)^2 \cdot \frac{1}{2} + (-1)^2 \cdot \frac{1}{2} = 1
  \]

  Hence,

  \[
    E\left[S_n\right] = \sum_{i=1}^n E\left[X_i\right] = 0
  \]

  \[
    Var\left[S_n\right] = \sum_{i=1}^n E\left[X_i^2\right] = n
  \]

  As \(n\)  approaches infinity,  the conditions  for the  \textit{Central Limit
  Theorem} hold: \(X_i\)s are \textit{i.i.d.}, so we can apply

  \[
    \frac{S_n - n \cdot E\left[X_i\right]}{\sqrt{n \cdot Var\left[X_i\right]}}
    = \frac{S_n}{\sqrt{n}} \underset{n \to \infty}{\sim} \mathcal{N}(0,1)
  \]

  Consider now a non-unitary  time. Suppose we move in time  steps of \(\delta >
  0\) and in space steps of \(\sqrt{\delta}\), and let's consider the process in
  the interval $[0,t],\ t\in\mathbb{R}^+$. Then,

  \[
    S_t = \sum_{i=0}^{\left\lfloor \frac{t}{\delta} \right\rfloor} X_i
  \]

  where     $X_i$     moves     by     $\pm\sqrt\delta$     with     probability
  $\mathbb{P}=\frac{1}{2}$. Then, $E\left[S_t\right]=0$ and

  \[
    Var[X_i] = E^2[X_i] - E[X_i^2] = E^2[X_i] = (+\sqrt{\delta})^2 \cdot \frac{1}{2} + (-\sqrt{\delta})^2 \cdot \frac{1}{2} 
      = \delta
  \]

  \[
    Var[S_t] = \frac{t}{\delta} \cdot \delta = t
  \]

  Let's now apply \textit{Central Limit Theorem}:

  \[
    \frac{S_t - \frac{t}{\delta} E[X_i]}{\sqrt{\frac{t}{\delta} Var[X_i]}}
    = \frac{S_t}{\sqrt{t}} \underset{t \to \infty}{\sim} \mathcal{N}(0,1).
  \]

  So, $S_t \sim \mathcal{N}(0,t)$: this is the \textit{Brownian motion}.

  \begin{defn} 
    \textbf{(Brownian motion).} The stochastic process 

    \[
      (W_t)_{t\in\mathbb{R}^+} : \mathbb{R}^+ \times (\Omega, (\mathcal{F}_t)_{t \geq 0}, \mathbb{P}) \to \mathbb{R}
    \]

    is a Brownian motion if and only if

    \begin{enumerate}
      \item $W_0 = 0$,
      \item it is continuous,
      \item has stationary increments: 
        
        $$\forall t > s,\ W_t - W_s \sim \mathcal{N}(0,t-s)$$

      \item has independent increments over disjoint intervals: 
        
        $$\forall q < r < s < t,\ (W_r-W_q) \perp (W_t-W_s).$$
    \end{enumerate}
  \end{defn}

  %\begin{enumerate}
  %  \item From Random Walk to Brownian Motion
  %  \item Brownian motion definition
  %  \item Classifications
  %\end{enumerate}

  \subsection{It\^o Formula}

\end{document}
